% Write your abstract here

This project presents the design and development of an intelligent energy monitoring and control system based on MODBUS communication protocol and Programmable Logic Controller (PLC). The aim of this study is to reduce unnecessary energy consumption, increase energy efficiency and ensure sustainable energy use in industrial environments by combining real-time monitoring, control algorithms and automation infrastructure.\vspace{3mm}

\medskip

The system collects real-time energy data via MODBUS from sensors integrated with machines such as DC motors and conveyor systems and transmits it to the PLC for analysis. Based on this data, control decisions are made using algorithms such as PID and LQR to optimize energy usage while maintaining system performance and safety. Factory IO software is used with Simulink and MATLAB to simulate the factory environment and validate the control strategy.

\medskip

Mathematical modeling and dynamic simulation of a brushless DC motor were performed to evaluate the energy behavior under different conditions. The model includes electrical and mechanical parameters and calculates torque, speed, efficiency, and energy consumption over time. Using a custom-built simulation setup, performance metrics such as motor efficiency and energy savings were analyzed.

\medskip

The project addresses critical engineering design criteria such as energy efficiency, system robustness, communication protocol integration and user safety. Engineering standards such as IEC 61158, IEC 61131-3 and ISO 50001 are implemented to ensure industrial compatibility and reliability.

\medskip

As a result, this project provides a viable and scalable solution for industrial energy optimization while also contributing to environmental sustainability and operational cost reduction.

\noindent\rule[2pt]{\textwidth}{0.5pt}

{\textbf{Keywords :}}
Automation, control, energy efficiency, industrial communication,  MODBUS, PLC, smart energy systems.\\
\noindent\rule[2pt]{\textwidth}{0.5pt}
