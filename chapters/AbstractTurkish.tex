% Write your abstract here

Bu proje, MODBUS iletişim protokolü ve Programlanabilir Mantık Denetleyicisi (PLC) tabanlı akıllı bir enerji izleme ve kontrol sisteminin tasarımını ve geliştirilmesini sunmaktadır. Bu çalışmanın amacı, gerçek zamanlı izleme, kontrol algoritmaları ve otomasyon altyapısını birleştirerek endüstriyel ortamlarda gereksiz enerji tüketimini azaltmak, enerji verimliliğini artırmak ve sürdürülebilir enerji kullanımını sağlamaktır.

\vspace{3mm}

Sistem, DC motorlar ve konveyör sistemleri gibi makinelerle entegre edilmiş sensörlerden MODBUS aracılığıyla gerçek zamanlı enerji verileri toplar ve bunları analiz için PLC'ye aktarır. Bu verilere dayanarak, sistem performansını ve güvenliğini korurken enerji kullanımını optimize etmek için PID ve LQR gibi algoritmalar kullanılarak kontrol kararları verilir. Fabrika ortamını simüle etmek ve kontrol stratejisini doğrulamak için Factory IO yazılımı, Simulink ve MATLAB ile birlikte kullanılır.

\vspace{3mm}

Farklı koşullar altında enerji davranışını değerlendirmek için fırçasız bir DC motorun matematiksel modellemesi ve dinamik simülasyonu gerçekleştirildi. Model, elektriksel ve mekanik parametreleri içerir ve zaman içinde torku, hızı, verimliliği ve enerji tüketimini hesaplar. Özel olarak oluşturulmuş bir simülasyon kurulumu kullanılarak, motor verimliliği ve enerji tasarrufu gibi performans ölçümleri analiz edildi.

\medskip

Proje, enerji verimliliği, sistem sağlamlığı, iletişim protokolü entegrasyonu ve kullanıcı güvenliği gibi kritik mühendislik tasarım kriterlerini ele alıyor. Endüstriyel uyumluluğu ve güvenilirliği sağlamak için IEC 61158, IEC 61131-3 ve ISO 50001 gibi mühendislik standartları uygulanmaktadır.

\medskip

Sonuç olarak, bu proje endüstriyel enerji optimizasyonu için uygulanabilir ve ölçeklenebilir bir çözüm sunarken aynı zamanda çevresel sürdürülebilirliğe ve operasyonel maliyet azaltımına da katkıda bulunuyor.\\[0.1mm]

\noindent\rule[2pt]{\textwidth}{0.5pt}

{\textbf{Anahtar Kelimeler :}}
akıllı enerji sistemleri, endüstriyel iletişim, enerji verimliliği, kontrol, MODBUS, otomasyon, PLC.
\\
\noindent\rule[2pt]{\textwidth}{0.5pt}




