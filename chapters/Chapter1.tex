\doublespacing % Do not change - required

\chapter{Introduction}
\label{ch1}

%%%%%%%%%%%%%%%%%%%%%%%%%%%%%%%%%%%%%%%
% IMPORTANT
\begin{spacing}{1} %THESE FOUR
\minitoc % LINES MUST APPEAR IN
\end{spacing} % EVERY
\thesisspacing % CHAPTER
% COPY THEM IN ANY NEW CHAPTER
%%%%%%%%%%%%%%%%%%%%%%%%%%%%%%%%%%%%%%%

\section{Motivation}

\begin{itemize}
    \item \textbf{Background and Context:}
    The increasing demand for energy efficiency, sustainability and cost optimization in industry has increased the interest in smart energy management solutions. One of the basic building blocks of these systems is the real-time collection of process data using communication protocols such as Modbus. Programmable Logic Controllers (PLCs) provide a reliable infrastructure for processing this data and implementing control algorithms. In this context, monitoring and optimizing energy consumption is of critical importance not only in reducing operational costs but also in contributing to environmental sustainability goals. This project aims to develop an innovative solution in the field of energy management by combining real-time data analysis and control algorithms.
    \item \textbf{Problem Identification:}
    Despite advances in automation, many industrial systems still face high operational costs and unnecessary environmental impacts due to inefficient energy use. Traditional energy management systems lack real-time data collection and advanced control mechanisms. This makes it difficult to detect abnormalities in energy consumption and to effectively execute optimization processes. Therefore, there is a need to develop intelligent systems that can perform real-time data analysis and optimize energy use using control algorithms such as PID, supported by Modbus-based communication. This project aims to provide a solution to the problem of inefficient energy management by developing an intelligent energy monitoring and control system using control algorithms. .
    \item \textbf{Significance of the Project:}
     This project integrates PLC infrastructure, communication protocols and real-time data analysis by developing a Modbus-based smart energy monitoring and control system. The aim is to optimize energy consumption, increase system stability and reduce operational costs by using real-time monitoring and control algorithms together. The solution developed in the study d offers a scalable structure that can be easily adapted to different industrial applications. Thus, it is aimed to make a significant contribution to energy efficiency and environmental sustainability goals.
    \item \textbf{Personal and Academic Motivation:}
    This study is directly compatible with our academic background in Control and Automation Engineering, combining theoretical knowledge with real-time monitoring, control algorithms and simulation-based system design. Through this project, we aim to develop our competencies in industrial communication, PLC programming, Modbus protocol usage and energy management. We also aim to create a strong infrastructure for our future career plans by contributing to global goals in the field of energy efficiency and sustainability.
\end{itemize}

\section{Design Criteria}
\begin{itemize}
    \item \textbf{Performance Requirements}
    \begin{itemize}
        \item The system must operate with accuracy to optimize energy consumption by at least 10.
        \item Real-time data monitoring time should be provided with a maximum delay of 1 second.
        \item The system must operate with over 95 stability and 98 fault tolerance.
    \end{itemize}

    \item \textbf{Technical Specifications}
    \begin{itemize}
        \item Hardware and software to be used: Siemens S7-1200 PLCSim, Modbus communication,MATLAB.
        \item Modbus
    \end{itemize}

    \item \textbf{Feasibility and Constraints}
    \begin{itemize}
        \item If implemented, the total system cost will be kept at a reasonable level.
        \item Power consumption will be minimized; data acquisition and control units will support low power modes.
        \item Hardware and software integration will be easy, making maintenance and expansion of the system practical.
    \end{itemize}

    \item \textbf{Scalability and Flexibility}
    \begin{itemize}
        \item The system will be modular and will allow for the integration of more sensors or different control algorithms in the future.
        \item It will be designed with flexibility to adapt to different production facilities and virtual factory environments.
    \end{itemize}

    \item \textbf{Safety and Reliability}
    \begin{itemize}
        \item Security measures (password protection, user authorization) will be applied to PLC communication.
        \item The system will be configured to switch itself to safe mode in case of communication interruptions.
        \item Hardware components will be tested periodically to ensure reliability.es
    \end{itemize}

    \item \textbf{Usability and Human Interaction}
    \begin{itemize}
        \item User-friendly monitoring and data analysis interfaces will be developed for operators.
        \item Easy and fast access to energy consumption and system status information will be provided.
    \end{itemize}

    \item \textbf{Environmental and Sustainability Considerations}
    \begin{itemize}
        \item Solutions that will increase energy efficiency will be preferred and carbon footprint will be minimized.
        \item If implemented, care will be taken to select recyclable materials in the components used.
    \end{itemize}
\end{itemize}
\section{Literature Review}
\begin{itemize}
\item \textbf{Previous Work:}
In the first examples of energy monitoring and control systems, manual data collection methods and local control systems were used. With the development of industrial automation, data transmission with Programmable Logic Controllers (PLCs) and Modbus protocol became widespread. Developed by Modicon in 1979, Modbus protocol has gained an important place in energy management applications by providing standard and reliable communication between different devices. Modbus RTU and TCP/IP based solutions have rapidly become widespread, especially for real-time monitoring of energy consumption and data transfer in factories.
\item \textbf{Related Studies:}
In recent studies, energy monitoring and control solutions developed with Modbus-based PLC systems are frequently used. For example, in the Sönmez study, applications were developed for monitoring and managing energy data with Modbus-supported automation systems. In the study conducted by Çetin, it was shown that energy efficiency could be increased by using control algorithms such as PID in motor systems. In addition, the ISO 50001 standard contributes to the continuous improvement-oriented structure of energy management systems.
\item \textbf{Research Gap}
When the existing studies are examined, it is seen that they mostly focus on monitoring energy data, and energy optimization applications with real-time control algorithms are limited. Although many systems perform the task of data collection, advanced control techniques for dynamic energy optimization are not integrated. This project aims to fill this gap by not only monitoring energy consumption, but also optimizing it by analyzing it in real time.
\end{itemize}

\section{Contribution of the Study}
\begin{itemize}
    \item \textbf{Scientific Contributions}
    \begin{itemize}
        \item A smart energy monitoring and control system integrated with Modbus communication has been designed.
        \item A PID-based control approach has been selected for real-time data collection and energy optimization.
        \item The developed system has been tested with comprehensive simulation studies based on Simulink.
    \end{itemize}

    \item \textbf{Technical Contributions}
    \begin{itemize}
        \item A real-time data acquisition and control platform was attempted to be created using Siemens S7-1200 PLCSim.
        \item A reliable communication infrastructure was attempted to be developed using TIA Portal software and Modbus protocol.
        \item System performance; accuracy, data transmission speed and stability were examined in terms of the system.
    \end{itemize}

    \item \textbf{Practical and Industrial Contributions}
    \begin{itemize}
        \item An attempt has been made to develop a system that can be applied in a virtual environment for industrial energy management.
        \item The system architecture has been designed to be easily adapted to a similar system in reality.   
        \item The developed solution serves the purpose of reducing energy costs and supporting sustainable production processes.
    \end{itemize}

    \item \textbf{Academic and Educational Contributions}
    \begin{itemize}
        \item A simulation-based work environment has been provided on control systems, industrial communication and energy management.
        \item Practical application experience has been gained on real-time data management, communication protocols and control algorithms.
       
    \end{itemize}

    \item \textbf{Societal and Environmental Contributions}
    \begin{itemize}
        \item It works to reduce the industrial carbon footprint by increasing energy efficiency.
        \item The project works towards environmental sustainability goals by optimizing resource use.
    \end{itemize}
\end{itemize}


\section{Structure of the Report}

This report is organized into several chapters, each covering different aspects of the study. A brief outline of each chapter is provided below:

\begin{itemize}
    \item \textbf{Chapter 1: Introduction}
    \begin{itemize}
        \item Provides background information and motivation for the study.
        \item Defines the research problem and objectives.
        \item Summarizes the structure of the thesis.
    \end{itemize}

    \item \textbf{Chapter 2: Literature Review}
    \begin{itemize}
        \item Reviews existing research and related works in the field.
        \item Identifies gaps and limitations in previous studies.
        \item Justifies the need for the proposed approach.
    \end{itemize}

    \item \textbf{Chapter 3: Methodology}
    \begin{itemize}
        \item Describes the theoretical foundations and mathematical models.
        \item Explains the proposed method, framework, or system design.
        \item Details the tools, software, and experimental setup used.
    \end{itemize}

    \item \textbf{Chapter 4: Implementation}
    \begin{itemize}
        \item Discusses the practical realization of the proposed method.
        \item Provides details on system architecture, hardware/software implementation.
        \item Highlights challenges and how they were addressed.
    \end{itemize}

    \item \textbf{Chapter 5: Results and Discussion}
    \begin{itemize}
        \item Presents experimental results, data analysis, and performance evaluation.
        \item Compares results with existing approaches.
        \item Discusses key findings and their implications.
    \end{itemize}

    \item \textbf{Chapter 6: Conclusion and Future Work}
    \begin{itemize}
        \item Summarizes the main contributions of the study.
        \item Discusses the limitations of the work.
        \item Suggests possible directions for future research and improvements.
    \end{itemize}

\end{itemize}


\clearpage