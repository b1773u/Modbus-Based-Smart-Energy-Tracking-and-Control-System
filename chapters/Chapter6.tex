\doublespacing % Do not change - required

\chapter{Conclusion and Future Work}
\label{ch6}

%%%%%%%%%%%%%%%%%%%%%%%%%%%%%%%%%%%%%%%
% IMPORTANT
\begin{spacing}{1} %THESE FOUR
\minitoc % LINES MUST APPEAR IN
\end{spacing} % EVERY
\thesisspacing % CHAPTER
% COPY THEM IN ANY NEW CHAPTER
%%%%%%%%%%%%%%%%%%%%%%%%%%%%%%%%%%%%%%%



    \section{Main Contributions of the Study}

Within the scope of this project, a real-time energy monitoring and control system integrated with a PLC using a Modbus-based communication protocol was developed. The modular architecture of the system allows for future expansion and improvement work. In addition, the design of the system using low-cost hardware that complies with industrial standards provided both economic efficiency and ease of application.
    \medskip

    \section{Limitations of The Work}

The system developed within the scope of the project was limited to a specific PLC model and energy analyzer. The integration of the system with different brands and models of devices was not tested. In addition, the communication infrastructure was implemented only via the Modbus RTU protocol, and compatibility with other communication protocols (such as Modbus TCP/IP or OPC UA) was not evaluated. Real-time performance analysis was performed only on specific sample load scenarios.    
    \medskip

    \section{Suggestions for Future Studies}

In future studies, the system can be integrated with different communication protocols to increase its flexibility. In addition, more in-depth analysis of energy consumption data can be made possible by adding advanced data analytics algorithms. Redesigning the system to work on different hardware platforms and integrating it with cloud-based monitoring systems will also enable the project outputs to be transferred to wider areas of use.
    \medskip




\clearpage